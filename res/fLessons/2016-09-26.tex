\section{Lezione 2016-09-26}
Definizione di \textbf{compilatore}:

\begin{definition}
I compilatori sono \textbf{processori di linguaggi}, traslano i programmi
scritti in un linguaggio in programma equivalente in un altro linguaggio.
\end{definition}

Il processo di compilazione è diviso in due parti:
\paragraph{Analisi}
Determina le operazione del programma, registrate in una struttura ad albero.
\paragraph{Sintesi}
Sposta le operazioni interne all'albero della fase precedente nel programma
obiettivo. \\

La progettazione di un compilatore è una \textbf{sfida}. L'evoluzione del
compilatore dipende dal linguaggio \textbf{sorgente e destinazione}:
\begin{itemize}
\item Integrazione algoritmi per supportare nuovi costrutti
\item Sfruttare l'architettura del computer altamente performante
\item Uso modello euristico per l'ottimalità del risultato
\end{itemize}

Il compilatore è diviso in molte parti software progettate mediante astrazione
ed l'applicazione di tecniche matematiche. L'uso del \textbf{giusto modello
matematico} garantirà l'uso del \textbf{corretto algoritmo}. Inoltre, bisogna
bilanciare la \textit{generalità} e \textit{potenza} contro l'
\textit{efficienza} e \textit{semplicità}.

La figura sottostante mostra le serie di passi compiuti dal compilatore per
convertire il codice sorgente nel codice \textit{target}:

\begin{figure}[H]
\begin{center}
\includegraphics[scale=0.4]{res/image/compiler_phases}
\caption{Passi del compilatore}
\label{fig:compiler_phases}
\end{center}
\end{figure}

I compilatori posso compiere \textbf{una} o \textbf{più volte} un insieme di
fasi di compilazione\footnote{Ma tutte o solo alcune?}. In base al numero delle
volte si definiscono:

\begin{definition}[Compilatore a passo singolo]
I compilatore a \textbf{passo singolo} eseguono una sola volta un insieme di
fasi di compilazione.
\end{definition}

\begin{definition}[Compilatore multi passo]
I compilatori \textbf{multi passo} eseguono più volte un insieme di fasi di
compilazione.
\end{definition}

Questo comportano una serie di contrapposizioni tra i due generi:

\begin{center}
\begin{table}{|l|l|}
  \hline

\end{table}
\end{center}
