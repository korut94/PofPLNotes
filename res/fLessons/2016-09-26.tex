\section{Lezione 2016-09-26}
\subsection{TODO}
\begin{table}[ht]
\begin{center}
\begin{tabular}{|p{\textwidth}|c|}
\hline
\multicolumn{1}{|c|}{\textbf{Miglioramento}} & \textbf{Sezione} \\ \hline
Capire meglio le varie attività svolte dal Front-End e Back-End &
\ref{sec:front-end} \\ \hline
Sezione Front-End non chiara, da integrare con la lettura del capito 2 [ALSU] &
\ref{sec:front-end} \\ \hline
Chiedere chiarimento ed aggiungere esempio linguaggio slide n.17 &
\ref{sec:gerarchia_chomsky} \\ \hline
Studiare le proprietà degli operatori nel libro [ALSU] cap.2 perchè non è troppo
chiaro e si va solo ad intuizione &
\ref{sec:proprieta_operatori} \\ \hline
L'esempio non è chiaro e non si capisce cosa intenda con ``Legati più
saldamente'' &
\ref{sec:proprieta_operatori} \\ \hline
\end{tabular}
\end{center}
\caption{Tabella miglioramenti}
\label{tab:tab_todo_2}
\end{table}

\subsection{Progettazione del compilatore}
\label{sec:progettazione_compilatore}
Definizione di \textbf{compilatore}:

\begin{definition}[\textbf{Compiler}]
I compilatori sono \textbf{processori di linguaggi}, traslano i programmi
scritti in un linguaggio in programma equivalente in un altro linguaggio.
\end{definition}

Il processo di compilazione è diviso in due parti:
\paragraph{Analisi}
Determina le operazione del programma, registrate in una struttura ad albero.
\paragraph{Sintesi}
Sposta le operazioni interne all'albero della fase precedente nel programma
obiettivo. \\

La progettazione di un compilatore è una \textbf{sfida}. L'evoluzione del
compilatore dipende dal linguaggio \textbf{sorgente e destinazione}:
\begin{itemize}
\item Integrazione algoritmi per supportare nuovi costrutti
\item Sfruttare l'architettura del computer altamente performante
\item Uso modello euristico per l'ottimalità del risultato
\end{itemize}

Il compilatore è diviso in molte parti software progettate mediante astrazione
ed l'applicazione di tecniche matematiche. L'uso del \textbf{giusto modello
matematico} garantirà l'uso del \textbf{corretto algoritmo}. Inoltre, bisogna
bilanciare la \textit{generalità} e \textit{potenza} contro l'
\textit{efficienza} e \textit{semplicità}.

La figura sottostante mostra le serie di passi compiuti dal compilatore per
convertire il codice sorgente nel codice \textit{target}:

\begin{figure}[H]
\begin{center}
\includegraphics[scale=0.4]{res/image/compiler_phases}
\caption{Passi del compilatore}
\label{fig:compiler_phases}
\end{center}
\end{figure}

\subsection{Tipi di compilatori}
\label{sec:tipi_compilatori}
I compilatori posso compiere \textbf{una} o \textbf{più volte} un insieme di
fasi di compilazione\footnote{Ma tutte o solo alcune?}. In base al numero delle
volte si definiscono:

\begin{definition}[\textbf{Single pass compilers}]
I compilatori a \textbf{passo singolo} eseguono una sola volta un insieme di
fasi di compilazione.
\end{definition}

\begin{definition}[\textbf{Multi pass compilers}]
I compilatori \textbf{multi passo} eseguono più volte un insieme di fasi di
compilazione.
\end{definition}

Questo comportano una serie di contrapposizioni tra i due generi:
\begin{table}[H]
\begin{center}
\begin{tabular}{p{3cm}|p{5cm}|p{5cm}|}
\cline{2-3}
& \multicolumn{2}{|c|}{Type compiler} \\ \cline{2-3}
& \multicolumn{1}{c|}{\textbf{Single pass}} &
\multicolumn{1}{c|}{\textbf{Multi pass}} \\ \hline
\multicolumn{1}{|c|}{\textbf{Efficiency}} &
More efficient & Usually slower \\ \hline
\multicolumn{1}{|c|}{\textbf{Memory}} & Use less memory &
Needs more memory \\ \hline
\multicolumn{1}{|c|}{\multirow{2}{*}{\textbf{Variable}}} &
Requires everthing to be defined before being used &
Declaration may be follow their use \\ \hline
\multicolumn{1}{|c|}{\multirow{2}{*}{\textbf{Feature}}} &
Influenced the design of early programming languages &
Allows better optmization of target code \\ \hline
\end{tabular}
\end{center}
\caption{
Tabella delle differenze tra compilatori \textit{single pass} e
\textit{multi pass}
}
\label{tab:difference_single_multi_pass_compilers}
\end{table}

\subsection{Il Front-End}
\label{sec:front-end}
La prima parte del compilatore va a definire la sintassi del linguaggio di
programmazione da implementare attraverso una \textit{Context-Free Grammars}.
Dato il programma in input questa prima parte andrà ad eseguire il parsing sia
\textbf{semantico} che \textbf{sintattico} attraverso l'analisi di prodotti
intermedi.

\begin{figure}[H]
  \includegraphics[scale=0.5]{res/image/front-end_structure}
  \caption{Struttura Front-End}
  \label{fig:front-end_structure}
\end{figure}

\subsection{Definizione sintassi}
\label{sec:definizione_sintassi}
\subsubsection{Grammatica}
\label{sec:grammatica}
\begin{definition}[\textbf{Grammar}]
Una \textbf{grammatica} è una 4-tupla $G = (N,T,P,S)$ dove:
\begin{itemize}
\item $T$ è un insieme finito di \textit{tokens} (simboli terminali)
\item $N$ è un insieme finito di \textbf{non} terminali
\item $P$ è un insieme finito di \textbf{produzioni} della forma
$$\alpha \longrightarrow \beta$$ dove $\alpha \in (N \cup T)^*N(N \cup T)^*$ e
$\beta \in (N \cup T)^*$
\item $S \in N$ è il simbolo iniziale designato
\end{itemize}
\label{def:grammatica}
\end{definition}

Con la scrittura $A^*$ si intente una \textbf{infinita} sequenza di elementi in
$A$. Se $A=\{a,b\} \Rightarrow A^*=\{\epsilon,a,b,aa,ab,ba,bb,aaa,...\}$. Per i
\textbf{non} terminali la sintassi $AB=\{ab \mid a \in A,b \in B\}$.

\subsubsection{Convenzioni d'uso notazioni}
\label{sec:convenzioni_notazioni}
\paragraph{Terminali}
$a,b,c,... \in T$, es. specifici terminali: $0,1,id,+$.
\paragraph{Non terminali}
$A,B,C,... \in N$, es. specifici non terminali: $expr, term, stmt$.
\paragraph{Simboli grammaticali}
$X,Y,Z \in (N{\cup}T)$.
\paragraph{Stringhe di terminali}
$u,v,w,x,y,z \in T^*$.
\paragraph{Stringhe simboli grammaticali}
$\alpha,\beta,\gamma \in (N{\cup}T)^*$.

\subsubsection{Derivazioni}
\label{sec:derivazioni}
\begin{definition}[\textbf{One-step Derivation}]
Un derivazione passo singolo è definita da
$$\gamma\alpha\delta \Rightarrow \gamma\beta\delta$$
dove $\alpha\to\beta$ è una produzione della grammatica.
\end{definition}

In aggiunta si definisce:
\begin{itemize}
\item $\Rightarrow \text{leftmost} \Rightarrow_{lm}$ se $\gamma$ \textbf{non}
contiene un \textit{non terminale}
\item $\Rightarrow \text{rightmost} \Rightarrow_{rm}$ se $\delta$ \textbf{non}
contiene un \textit{non terminale}.
\item Chiusura transitiva $\Rightarrow^*$ (zero o più passi)
\item Chiusura positiva $\Rightarrow^+$ (uno o più passi)
\end{itemize}

\begin{definition}[\textbf{Sentential form}]
$\alpha$ di definisce una forma sentenziale se $S \Rightarrow^* \alpha$.
\end{definition}

\begin{definition}[\textbf{Language generated}]
Sia $G$ una grammatica, allora si definisce il linguaggio generato da $G$ come
$$L(G)=\{w \in T^* \mid S \Rightarrow^+ w\}$$
\end{definition}

\begin{proof}[Esempio derivazione 1]
Data la grammatica $G=(\{E\},\{+,*,(,),-,id\},P,E)$ con la produzione
\begin{align*}
P = \quad
& E \to E \textbf{+} E \\
& E \to E \textbf{*} E \\
& E \to \textbf{(}E\textbf{)} \\
& E \to \textbf{-} E \\
& E \to \textbf{id}
\end{align*}
esempi di derivazione possono essere:
\begin{align}
& E \Rightarrow \textbf{-} E \Rightarrow \textbf{-id} \\
& E \Rightarrow_{rm} E \textbf{+} E \Rightarrow_{rm} E \textbf{+id}
\Rightarrow_{rm} \textbf{id+id} \\
& E \Rightarrow^* E \\
& E \Rightarrow^* \textbf{id+id} \\
& E \Rightarrow^+ \textbf{id*id+id}
\end{align}
\end{proof}

\begin{proof}[Esempio derivazione 2]
Prendiamo un'altra grammatica
$$G=(\{list,digit\},\{+,-,0,1,2,3,4,5,6,7,8,9\},P,list)$$ dove la produzione $P$
è
\begin{align*}
P =  \quad
& list \to list \textbf{+} list \\
& list \to list \textbf{-} digit \\
& list \to digit \\
& digit \to \textbf{0}|\textbf{1}|\textbf{2}|\textbf{3}|\textbf{4}|\textbf{5}|
\textbf{6}|\textbf{7}|\textbf{8}|\textbf{9}
\end{align*}
una derivazione \textit{leftmost} è come segue:
\begin{align*}
& \underline{list} \\
& \Rightarrow_{lm} \underline{list} \textbf{+} digit \\
& \Rightarrow_{lm} \underline{list} \textbf{-} digit \textbf{+} digit \\
& \Rightarrow_{lm} \underline{digit} \textbf{-} digit \textbf{+} digit \\
& \Rightarrow_{lm} 9 \textbf{-} \underline{digit} \textbf{+} digit \\
& \Rightarrow_{lm} 9 \textbf{-} 5 \textbf{+} \underline{digit} \\
& \Rightarrow_{lm} 9 \textbf{-} 5 \textbf{+} 2
\end{align*}
\end{proof}

\subsubsection{Classificazione dei linguaggi}
\label{sec:classificazione_linguaggi}
A seconda delle sue produzioni, una grammatica $G$ è detta:
\paragraph{Regolare (\textbf{Regular})}
se è \textit{right linear} dove ogni produzione è della forma
$$A \to wB \quad \text{o} \quad A \to w$$
o \textit{left linear} dove ogni produzione è della forma
$$A \to Bw \quad \text{o} \quad A \to w$$
dove $w \in T^*$.
\paragraph{Libere dal contesto (\textbf{Context free})}
se ogni produzione è della forma $$A \to \alpha$$ dove $A \in N$ e $\alpha \in
(N \cup T)^*$.
\paragraph{Sensibile al constesto (\textbf{Context sensitive})}
se ogni produzione è della forma $$\alpha A \beta \to \alpha \gamma \beta$$
dove $A \in N, \alpha,\gamma,\beta \in (N \cup T)^*, |\gamma| > 0$.
\paragraph{Illimitato (\textbf{Unrestricted})}

\subsubsection{Gerarchia di Chomsky}
\label{sec:gerarchia_chomsky}
\textit{Chomsky} definisce la relazione d'inclusione fra i vari linguaggi:
$$\mathcal{L}(\text{regular}) \subset
\mathcal{L}(\text{context free}) \subset
\mathcal{L}(\text{context sensitive}) \subset
\mathcal{L}(\text{unrestricted})$$
dove $\mathcal{L}(T) = \{L(G) \mid G$ è il tipo di $T\}$, ovvero l'insieme di
\textbf{tutti} i linguaggi generati dalla grammatica $G$ del tipo $T$.

\subsubsection{Albero sintattico}
\label{sec:albero_sintattico}
La derivazione viene mostrata come un \textbf{albero}\footnote{si veda
\url{https://www.di.unipi.it/~andrea/Didattica/PLP-16/SLIDES/PLP-2016-03.pdf}
slide n.19 per esempio}(\textit{Parse Tree}) con le seguenti caratteristiche:
\begin{enumerate}
\item La \textit{root} dell'albero è etichettata da un \textit{terminale} o
da $\epsilon$
\item Ogni nodo interno è etichettato con un \textit{non-terminale}
\item Se $A \to X_1 X_2 ... X_n$ è una produzione, allora il nodo ha
immediatamente i figli $X_1, X_2, ..., X_n$ dove $X_i$ è un
\textit{(non-)terminale} o $\epsilon$
\end{enumerate}

\begin{definition}[Ambiguity]
Data una grammatica $G=(N,T,P,S)$ si dice che è \textbf{ambigua} se esistono
stringhe da essa generate diverse, o equivalenti, che hanno più di un possibile
albero sintattico.
\end{definition}

\subsubsection{Proprietà operatori}
\label{sec:proprieta_operatori}
Gli operatori possono godere della proprietà \textbf{associativa}:

\paragraph{Left-associative}
Operatori \textit{left-associative} hanno produzioni \textbf{ricorsive sinistre}
del tipo $$left \to left \textbf{+} term | term$$ ad esempio la stringa
$\textbf{a+b+c}$ ha lo stesso significato di $\textbf{(a+b)+c}$.

\paragraph{Right-associative}
Operatori \textit{right-associative} hanno produzioni \textbf{ricorsive destre}
del tipo $$right \to term \textbf{=} right | term$$ ad esempio la stringa
$\textbf{a=b=c}$ ha lo stesso significato di $\textbf{a=(b=c)}$. \\

Inoltre possono godere della proprietà di \textbf{precedenza}, dove operatori
con \textbf{maggior} precedenza sono ``\textbf{legati molto più saldamente}
\footnote{originale diceva ``bind more tightly''}'':
\begin{align*}
& expr \to expr\textbf{+}term|term \\
& term \to term\textbf{*}factor|factor \\
& factor \to number|\textbf{(}expr\textbf{)}
\end{align*}
