\section{Lezione 2016-09-26}
\subsection{TODO}
\begin{table}[ht]
\begin{center}
\begin{tabular}{|l|c|}
\hline
\multicolumn{1}{|c|}{\textbf{Miglioramento}} & \textbf{Sezione} \\ \hline
Capire meglio le varie attività svolte dal Front-End e Back-End &
\ref{sec:front-end} \\ \hline
Sezione Front-End non chiara, da integrare con la lettura del capito 2 [ALSU] &
\ref{sec:front-end} \\ \hline
\end{tabular}
\end{center}
\caption{Tabella miglioramenti}
\label{tab:tab_todo_2}
\end{table}

\subsection{Progettazione del compilatore}
\label{sec:progettazione_compilatore}
Definizione di \textbf{compilatore}:

\begin{definition}
I compilatori sono \textbf{processori di linguaggi}, traslano i programmi
scritti in un linguaggio in programma equivalente in un altro linguaggio.
\end{definition}

Il processo di compilazione è diviso in due parti:
\paragraph{Analisi}
Determina le operazione del programma, registrate in una struttura ad albero.
\paragraph{Sintesi}
Sposta le operazioni interne all'albero della fase precedente nel programma
obiettivo. \\

La progettazione di un compilatore è una \textbf{sfida}. L'evoluzione del
compilatore dipende dal linguaggio \textbf{sorgente e destinazione}:
\begin{itemize}
\item Integrazione algoritmi per supportare nuovi costrutti
\item Sfruttare l'architettura del computer altamente performante
\item Uso modello euristico per l'ottimalità del risultato
\end{itemize}

Il compilatore è diviso in molte parti software progettate mediante astrazione
ed l'applicazione di tecniche matematiche. L'uso del \textbf{giusto modello
matematico} garantirà l'uso del \textbf{corretto algoritmo}. Inoltre, bisogna
bilanciare la \textit{generalità} e \textit{potenza} contro l'
\textit{efficienza} e \textit{semplicità}.

La figura sottostante mostra le serie di passi compiuti dal compilatore per
convertire il codice sorgente nel codice \textit{target}:

\begin{figure}[H]
\begin{center}
\includegraphics[scale=0.4]{res/image/compiler_phases}
\caption{Passi del compilatore}
\label{fig:compiler_phases}
\end{center}
\end{figure}

\subsection{Tipi di compilatori}
\label{sec:tipi_compilatori}
I compilatori posso compiere \textbf{una} o \textbf{più volte} un insieme di
fasi di compilazione\footnote{Ma tutte o solo alcune?}. In base al numero delle
volte si definiscono:

\begin{definition}[Compilatore a passo singolo]
I compilatore a \textbf{passo singolo} eseguono una sola volta un insieme di
fasi di compilazione.
\end{definition}

\begin{definition}[Compilatore multi passo]
I compilatori \textbf{multi passo} eseguono più volte un insieme di fasi di
compilazione.
\end{definition}

Questo comportano una serie di contrapposizioni tra i due generi:
\begin{table}[H]
\begin{center}
\begin{tabular}{p{3cm}|p{5cm}|p{5cm}|}
\cline{2-3}
& \multicolumn{2}{|c|}{Type compiler} \\ \cline{2-3}
& \multicolumn{1}{c|}{\textbf{Single pass}} &
\multicolumn{1}{c|}{\textbf{Multi pass}} \\ \hline
\multicolumn{1}{|c|}{\textbf{Efficiency}} &
More efficient & Usually slower \\ \hline
\multicolumn{1}{|c|}{\textbf{Memory}} & Use less memory &
Needs more memory \\ \hline
\multicolumn{1}{|c|}{\multirow{2}{*}{\textbf{Variable}}} &
Requires everthing to be defined before being used &
Declaration may be follow their use \\ \hline
\multicolumn{1}{|c|}{\multirow{2}{*}{\textbf{Feature}}} &
Influenced the design of early programming languages &
Allows better optmization of target code \\ \hline
\end{tabular}
\end{center}
\caption{
Tabella delle differenze tra compilatori \textit{single pass} e
\textit{multi pass}
}
\label{tab:difference_single_multi_pass_compilers}
\end{table}

\subsection{Il Front-End}
\label{sec:front-end}
La prima parte del compilatore va a definire la sintassi del linguaggio di
programmazione da implementare attraverso una \textit{Context-Free Grammars}.
Dato il programma in input questa prima parte andrà ad eseguire il parsing sia
\textbf{semantico} che \textbf{sintattico} attraverso l'analisi di prodotti
intermedi.

\begin{figure}[H]
  \includegraphics[scale=0.5]{res/image/front-end_structure}
  \caption{Struttura Front-End}
  \label{fig:front-end_structure}
\end{figure}

\subsection{Definizione sintassi}
\label{sec:definizione_sintassi}
\subsubsection{Grammatica}
\label{sec:grammatica}
\begin{definition}
Una \textbf{grammatica}
\end{definition}
