\section{Lezione 2016-09-29}
\subsection{TODO}
% Insert what you need. Any row is associated with the improvment or mistake
% arise. In the first column you can insert what you should resolve or change,
% instead in the second column you may put the section where to apply some
% modification.
\begin{table}[ht]
\begin{center}
\begin{tabular}{|p{\textwidth}|c|}
\hline
\multicolumn{1}{|c|}{\textbf{Miglioramento}} & \textbf{Sezione} \\ \hline
Effettuare una descrizione più precisa dell'obiettivo del
\textit{Sytax-Directed Translation} &
\ref{sec:syntax-directed_translation}\\ \hline
Descrivere il funzionamento del \textit{lookahead token} &
\ref{sec:parsing_predittivo} \\ \hline
Aggiungere un esempio del \textit{lookahead token} &
\ref{sec:parsing_predittivo} \\ \hline
Chiedere come si risolverebbero i conflitti nel caso di più identificatori nel
programma & \ref{soltwo} \\ \hline
Il parser come gestirebbe i conflitti & \ref{solthree} \\ \hline
Cosa serve produrre codice intermedio
& \ref{sec:generatore_codice_intermedio} \\ \hline
Capire in cosa consiste lo schema di traslazione &
\ref{sec:schema_traslazione} \\ \hline
Aggiungere sezione riguardante le tabelle di traslazione, davvero poco chiare
le slide & \ref{sec:ast} \\ \hline
\end{tabular}
\end{center}
\caption{Tabella miglioramenti}
\label{tab:tab_todo}
\end{table}

\subsection{Sytax-Directed Translation}
\label{sec:syntax-directed_translation}
Modulo del \textit{Front-End} del compilatore che attraverso una grammatica
libera dal contesto permette di specificare la \textbf{struttura sintattica} del
linguaggio. Inoltre ha il compito di:
\begin{enumerate}
\item associare ai terminali e non della grammatica un \textbf{insieme di
attributi}
\item associare ad ogni produzione un \textbf{insieme di regole semantiche} per
calcolare i valori degli attributi
\end{enumerate}

Attraverso l'attraverso dell'albero sintattico vengono applicate le regole
semantiche: una volta finito l'attraversamento, i valori degli attributi sui non
terminali contengono la \textbf{forma traslata dell'input}. Nota: può essere
usato per l'analisi sintattica e semantica statica.

Un possibile attraversamento può essere il \textit{Depth-First} dove si vanno
ad analizzare prima i figli e poi i genitori. L'algoritmo è il seguente:

\begin{lstlisting}[frame=single]
procedure visit(n:node);
begin
  for each child m of n, from left to right do
    visit(m);
    evalueate semantics rules at the node n
end
\end{lstlisting}

Le regole semantiche vengono definite insieme alla produzioni per la
composizione dell'albero sintattico. Un esempio potrebbe essere:
\begin{align*}
expr &\to expr_1 + term  & expr.t &:= expr_1.t // term.t // ``\textbf{+}''  \\
expr &\to expr_1 - term  & expr.t &:= expr_1.t // term.t // ``\textbf{-}''  \\
expr &\to term           & expr.t &:= term.t                                \\
expr &\to \textbf{0}     & term.t &:= ``\textbf{0}''                        \\
expr &\to \textbf{1}     & term.t &:= ``\textbf{1}''                        \\
     &...                & ...                                              \\
expr &\to \textbf{9}     & term.t &:= ``\textbf{9}''
\end{align*}

che data la stringa in input va a generare la forma \textbf{postfissa} dell'
espressione. La colonna di sinistra sono le varie produzioni, mentre quella a
destra sono le regole semantiche.

Gli attributi valutati nei nodi dell'albero possono essere chiamati:

\paragraph{Sintetico (\textit{Synthesized})}
se il valore del nodo è determinato dai valori degli attributi dei nodi figli.
\paragraph{Ereditato (\textit{Inherited})}
se il valore del nodo è determinato dal genitore. Se sono presenti attributi di
questo tipo l'attraversamento \textit{Depth-First} non funziona.

\subsection{Schema di traslazione}
\label{sec:schema_traslazione}
\begin{definition}[Translation Schemes]
Uno \textit{schema di traslazione} è una \textit{Context-free grammar}
incorporata con le \textbf{azioni semantiche}
\end{definition}

Queste azioni semantiche vengono viste come \textbf{simboli terminali} del
nostro albero sintattico. L'idea è che durante l'attraversamento
\textit{Depth-First} o \textit{Left-To-Right} essi vengono eseguiti. Per l'
esempio di prima il \textit{Transition Scheme} per la notazione postfissa
risulta:

\begin{proof}[Esempio schema di transizione]
\begin{align*}
& expr \to expr \textbf{+} expr & \{print(``+'')\} \\
& expr \to expr \textbf{-} expr & \{print(``-'')\} \\
& expr \to term                                    \\
& term \to \textbf{0}           & \{print(``0'')\} \\
& term \to \textbf{1}           & \{print(``1'')\} \\
& ...                                              \\
& term \to \textbf{9}           & \{print(``9'')\} \\
\end{align*}
\end{proof}

\subsection{Parsing}
\begin{definition}[Parsing]
Il parsing è un \textbf{processo} per determinare se una stringa di
\textit{token} possono essere generati da una grammatica.
\end{definition}

Il processo di parsing va a ``costruire'' un albero sintattico a partire:
\begin{itemize}
\item \textbf{dalla radice alle foglie} se è un parsing \textit{top-down}
\item \textbf{dalle foglie alla radice} se è un parsing \textit{bottom-up}
\end{itemize}

Vi è un teorema interessante la quale dice che \textbf{per ogni} grammatica
libera dal contesto \textbf{c'è} un parsing che prende \textbf{al più} $O(n^3)$
per eseguire il parsing di una stringa di $n$ \textit{token}.

\subsection{Parsing predittivo}
\label{sec:parsing_predittivo}
Il parsing predittivo fa parte della famiglia dei parser \textit{top-down}
derivato dal \textit{Recursive Descent Parsing}, dove:
\begin{itemize}
\item ogni non-terminale ha una produzione che è responsabile di eseguire il
parsing della categoria sintattica\footnote{Cos'è?} dei non-terminali di input
\textit{token}.
\item quando un non-terminale ha produzioni multiple, vengono prodotte una per
una finchè una non ha successo. Se tutte falliscono, la procedura per il
non-terminale fallisce.
\item \textit{backtracking} è necessario e la complessità è
\textbf{esponenziale}
\end{itemize}

Il \textit{parsing predittivo} deriva dal \textit{Recursive Descent Parsing}
sfruttando i \textit{lookahead tokens}\footnote{Come sono fatti?} per
inequivocabilmente determinare la produzione da provare. La complessità è
\textbf{lineare}. L'idea è seguire un controllo guidato a partire da un token
e si va a controllare se è il valore aspettato:
\begin{itemize}
\item se è quello atteso si passa al successivo finché non terminato
\item se è errato si lancia un errore
\end{itemize}

\subsection{First}
\begin{definition}[First]
$FIRST(\alpha)$ è l'insieme dei terminali che appare come \textbf{primo}
simbolo di uno o più stringhe generate da $\alpha$.
\end{definition}

Utilizzando il \textit{FIRST} si migliora l'algoritmo del parsing predittivo
sostituendo nei branch decisionali (relativi alle produzioni) la condizione
d'ingresso. Al posto di inserire manualmente l'insieme dei primi terminali ci
pensa il \textit{FIRST} della produzione ad individuarli.

Il problema è che quando un non terminale ($A$) ha due o più produzionie
($\alpha$ e $\beta$) allora gli insiemi risultanti dal \textit{FIRST} di ogni
produzione devono essere \textbf{disgiunti}
($FIRST(\alpha) \cap FIRST(\beta) = \emptyset$) perché l'algoritmo possa
funzionare.

Attraverso l'utilizzo del \textit{Left Factoring} è possibile convertire una
grammatica $G$ in $G'$ equivalenti dove le produzioni di $G'$ non cominciano con
lo stesso simbolo, ovvero i \textit{FIRST} delle produzioni sono disgiunte.

\begin{proof}[Esempio]
Data la grammatica $G = (N,T,P,S)$ con $P$
\begin{align*}
  stmt \to
  & \ \text{\textbf{if}} \ expr \ \text{\textbf{then}} \
    stmt \ \text{\textbf{endif}} \\
  & \mid \text{\textbf{if}} \ expr \ \text{\textbf{then}} \ stmt \
    \text{\textbf{else}} \ stmt \ \text{\textbf{endif}}
\end{align*}

possiamo usare il \textit{Left Factoring} per modificare $P$ in $P'$ risolvendo
il problema degli insiemi disgiunti:
\begin{align*}
& stmt \to \ \text{\textbf{if}} \ expr \ \text{\textbf{then}} \ stmt \
             opt\_else \\
& opt\_else \to \text{\textbf{else}} \ stmt \ \text{\textbf{endif}}
                \mid \text{\textbf{endif}}
\end{align*}
\end{proof}

Altro elemento di disturbo è la \textit{Left Recursion}, nel caso di un non
terminale $A$ con una produzione del tipo
\begin{align*}
A \to & A \alpha \\
      & \mid \beta \\
      & \mid \gamma
\end{align*}

un parser predittivo andrebbe in \textit{loop}. Possiamo eliminare le
\textbf{produzioni ricorsive a sinistra} riscrivendo la grammatica usando
\textbf{produzioni ricorsive a destra}, come segue:
\begin{align*}
& A \to \beta R \mid \gamma R  \\
& R \to \alpha R \mid \epsilon
\end{align*}

\subsection{Analizzatore lessicale}
\label{sec:analizzatore_lessicale}
Chiamato anche \textit{lexer} i suoi tipici task da svolgere sono:
\begin{itemize}
\item Rimuovere spazi bianchi e commenti
\item Codificare costanti come \textit{token}
\item Riconoscere le \textit{keyword}
\item Riconoscere identificatori e immagazzinare il nome in una
\textbf{tabella dei simboli}(\textit{symobol table}) globale
\end{itemize}

Dopo l'analisi della stringa in input, viene generata una sequenza di coppie
$<$\textbf{token}, \textbf{tokenval}$>$ dove il primo rappresenta il
\textit{lookahead} mentre il secondo il valore riscontrato nell'input.

Tramite le variabili globali \textbf{lookahead} e \textbf{tokenval} nel codice
si accede al token attulmente computato\footnote{Vedere slide 33 per esempio}.
I token vengono salvati nella tabella dei simboli attraverso la struttura dati:

\begin{lstlisting}[language=C]
struct entry
{
  char *lexptr; /*lexema (stringa) per tokenval*/
  int token;
}

struct entry symtable[];

// Ritorna l'indice della nuova entry per la stringa
// inserita (valore attuale) e il token a cui
// appartiene
insert(s, t)
// Ritorna l'indice della entry per la stringa ``s''
lookup(s)
\end{lstlisting}

Ogni linguaggio di programmazione è composto dagli \textbf{identificatori}
(\textit{identifiers}) formulati dal programmatore e le
\textbf{parole riservate}(\textit{reserved keywords}) proprie del linguaggio.
Ogninua di esse va gestito nel \textit{lexer} per generare i token e nel
\textit{parser} per ottenere la semantica del programma.

\subsubsection{Gestione identificatori}
In un ipotetico analizzatore lessicale scritto in \textit{C}
\paragraph{Nel lexer}
Codice che viene eseguito una volta trovato un identificatore. La stringa viene
salvata nella variabile \textbf{lextbuf}

\begin{lstlisting}[language=C]
int lexan()
{
  // ...

  // Ritorna l'indice dell'identificatore in modo da
  // controllare se era gia' presente
  tokenval = lookup(lexbuf);
  // Se l'identificatore non e' mai stato
  // inserito...
  if (tokenval == 0) {
    // ... si aggiunge una entry alla tabella con la
    // coppia <ID, tokenbuf> ottenendo la posizione
    // attuale
    tokenval = insert(lexbuf, ID);
  }

  return symtable[tokenval].token;
}
\end{lstlisting}

\paragraph{Nel parser}
Esecuzione delle produzioni al fine di ottenere la semantica del programma
\begin{lstlisting}[language=C]
#define ID 259 /* Token restituito dal lexer */

void factor()
{
  if (lookahead == '(') {
    match('('); expr(); match('(');
  } else if (lookahead == ID) {
    // Tokenval e' una variabile globale configurata
    // nel lexer
    printf("%s", symtable[tokenval].lexprt);
    match(ID);
  } else {
    error();
  }
}
\end{lstlisting}

\subsubsection{Gestione keyword riservate}
Per rendere effettive le \textit{keyword} riservate è sufficiente inserirle
nella tabella globale dei simboli all'inizializzazione del compilatore, prima
di ogni analisi sul codice in input.

\paragraph{Nel lexer}
Vanno distinti tre parti:
\begin{itemize}
\item I file con i token assegnati alle \textit
\item Inserimento nella tabella dei simboli
\item La consueta procedura d'analisi
\end{itemize}

Rispettivamente vengono creati tre file (\textit{global.h}, \textit{init.c},
\textit{lexer.c}) dove eseguire le tre procedure. Il lexer rimane invariato
mentre gli altri due seguono la forma:
\begin{lstlisting}[language=C,caption=global.h]
// Lista keyword per il linguaggio
#define DIV 257 /* token */
#define MOD 258 /* token */
#define ID  259 /* token */
\end{lstlisting}

\begin{lstlisting}[language=C,caption=init.c]
#include "global.h"

// Lista inserimenti nella tabella dei simboli
insert("div", DIV);
insert("mod", MOD);
\end{lstlisting}

In un programma lo stesso identificatore (es. variabile) può essere utilizzato
più volte, tipo in diversi blocchi del programma, tuttavia potrebbero assumere
significati completamente diversi (es. cambio del tipo) modificando gli
attributi del token associato. Con l'algoritmo attuale si andrebbe a generare
un conflitto, tenendo per buono il primo identificatore scoperto. Per risolvere
ciò una soluzione è usare un'\textbf{unica} tabella dei simboli
\textbf{per ogni scope}. La gestione delle varie tabella potrebbe essere:
\begin{enumerate}
\item concatenarle per blocchi innestati
\item tabella hash + stack ausiliario(?)\label{soltwo}
\item le entry della tabella vengono create dal parser(?)\label{solthree}
\end{enumerate}

\subsection{Generatore codice intermedio}
\label{sec:generatore_codice_intermedio}
Ci sono due tipologie di codice intermedio che portano vari vantaggi:
\paragraph{Alberi (Trees)}
molto utile per \textbf{analisi semantica}("\textbf{static checking}")
\begin{itemize}
\item \textit{parse trees}
\item \textit{abstract syntax trees}
\end{itemize}

\paragraph{Rappresentazioni lineari (Linear rappresentation)}
buona per ottimizzazione del codice indipendente dalla macchina
\begin{itemize}
\item \textit{three-address code}
\end{itemize}

\subsubsection{Abstract Syntax Trees}
\label{sec:ast}
A differenza dell'albero sintattico, la quale rappresenta la
\textbf{sintassi concreta} del programma, l'\textit{Abstract Syntax Trees} o
\textbf{AST} è la rappresentazione ad albero della \textbf{sintassi astratta}
del programma; ovvero:
\begin{itemize}
\item rimosse le parentesi
\item eliminate le \textit{keyword}
\item construtti rappresentati dai nodi
\end{itemize}

\subsection{Controllo statico}
Le proprietà sintattiche (non catturate dal grammatica del linguaggio) sono
controllate attraverso l'analisi dal \textit{parse tree} e
\textit{abstract syntax tree}

Le proprietà sintattica \textit{context-dependent} sono:
\begin{enumerate}
\item ogni variabile è dichiarata dopo l'uso
\item ogni identificatore è dichiarato al più una volta per scope
\item operandi sinistri dell'assegnazione sono \textit{L-valori}
\item l'operatore \textit{break} deve essere racchiuso in un loop o switch
\end{enumerate}

\subsection{Analisi semantica}
\begin{definition}
L'\textbf{analisi semantica} è applica dal compilatore per scoprire il
significato di un programma analizzando il suo \textbf{parse tree} o
\textbf{abstract syntax tree}. Molto utile per prevenire
\textbf{errori runtimes}.
\end{definition}

L'operazione d'analisi semantica può essere eseguita o a \textit{static} o
\textit{dynamic time}. Nel secondo caso è il compilatore a produrre codice che
va effettivamente ad eseguire l'analisi. In base a quando eseguire i controlli
si possono eseguire vari riconoscimenti:
\paragraph{Static semantic checks}
\begin{itemize}
\item Type checking, ogni operatore è applicato all'argomento del
giusto tipo
\item Handling coercion and overloading
\end{itemize}

\paragraph{Dynamic semantic checks}
\begin{itemize}
\item Indice array all'interno dei limiti
(\textit{subscript values within bounds})
\item Errori aritmetici, divisione per 0
\item Puntatori non deferenziati a meno che non puntino ad oggetti validi
\item Variabili usate ma mai inizializzate
\item Quando un controllo fallisce viene lanciata un'eccezione
\end{itemize}

\subsection{Generazione Three Address Code}
Il three-address code è una rappresentazione lineare da induzione strutturale
eseguendo una funzione sui nodi dell'albero.
\paragraph{Struttura del linguaggio}
\begin{itemize}
\item Sequenza di istruzioni della forma $$x = y \ op \ z$$
\item Array gestiti con istruzioni
\begin{align*}
& x[y] = z \\
& x = y[z]
\end{align*}
\item Sequenze di controllo gestite con istruzioni \textbf{jump}
\begin{align*}
& ifFalse \ x \ goto \ L \\
& ifTrue \ x \ goto \ L \\
& goto \ L
\end{align*}
\item Le istruzioni possono avere qualunque numero d'etichette
$$L1:L2: \ x = y$$
\end{itemize}

\paragraph{Traslazione di espressioni}
\begin{itemize}
\item Un'operazione alla volta, con l'utilizzo dei \textbf{temporanei}
$$
i - j + k \to \left\{
\begin{array}{lr}
  t1 = & i - j \\
  t2 = & t1 + k
\end{array}
\right.
$$
$$
2 * a[i] \to \left\{
\begin{array}{lr}
  t1 = & a[i] \\
  t2 = & 2 * t1
\end{array}
\right.
$$
\item L-valori in assegnazione non possono essere traslati in temporanei
$$
a[i] = 2 * a[j-k] \to \left\{
\begin{array}{lr}
  t3   = & j - k \\
  t2   = & a[t3] \\
  t1   = & 2 * t2 \\
  a[i] = & t1
\end{array}
\right.
$$
\end{itemize}

Vedi slide
\url{https://www.di.unipi.it/~andrea/Didattica/PLP-16/SLIDES/PLP-2016-04.pdf}
n.49 per un esempio di come vedere un generatore orientato agli oggetti.
