\section{Lezione 2016-10-03}
\subsection{TODO}
% Insert what you need. Any row is associated with the improvment or mistake
% arise. In the first column you can insert what you should resolve or change,
% instead in the second column you may put the section where to apply some
% modification.
\begin{table}[ht]
\begin{center}
\begin{tabular}{|p{\textwidth}|c|}
\hline
\multicolumn{1}{|c|}{\textbf{Miglioramento}} & \textbf{Sezione} \\ \hline
Chiedere specifiche perché realizzare un generatore di Lexical Analyzer
& \ref{sec:lexer_generatore} \\ \hline
Risolvere conflitti cosa intente? & \ref{sec:lexer_generatore} \\ \hline
Comprendere il senso della risoluzione dei conflitti &
\ref{sec:costruzione_thompson} \\ \hline
Chiedere chiarimento su cosa siano \textit{Dstates} e \textit{Dtran} e cosa si
intenda con \textit{unmarked state} nell'algoritmo di conversione NFA-DFA &
\ref{sec:NFA_to_DFA} \\ \hline
\end{tabular}
\end{center}
\caption{Tabella miglioramenti}
\label{tab:tab_todo}
\end{table}

\subsection{Progettazione di un generatore di lexer}
\label{sec:lexer_generatore}
La progettazione di lexer che dato in input un linguaggio produce i token per il
parse richiede una serie di passaggi:
\begin{enumerate}
\item Dalla RE (\textit{regular expression}) si costruisce un NFA
(non-deterministic finite automaton) che accetta lo stesso linguaggio regolare
\item Combinare gli NFA in uno singolo
\item E' possibile \textbf{simulare direttamente} l'NFA o una volta determinato
l'NFA simulare il DFA (deterministic finite automaton) risultante
\item Risolvere conflitti
\end{enumerate}

\subsubsection{Non-deterministic Finite Automata}
\begin{definition}[Non-deterministic Finite Automata - NFA]
Un NFA è una 5-tupla $(S,\Sigma,\delta,s_0,F)$ dove:
\begin{itemize}
\item S è un insieme finito di stati
\item $\Sigma$ è un insieme finito di simboli, l'alfabeto
\item $\delta$ è una mappa da $S \times (\Sigma \cup \{\epsilon\})$ ad
un'insieme di stati
$$\delta : S \times (\Sigma \cup \{\epsilon\}) \to 2^S$$
\item $s_0 \in S$ è lo stato di partenza
\item $F \subseteq S$ è l'insieme degli stati accettanti (o finali)
\end{itemize}
\end{definition}

La rappresentazione del NFA può avvenire mediante grafo di transizione,
grafo orientato etichettato

\begin{figure}[H]
\includegraphics[scale=0.5]{res/image/nfa}
\caption{Grafo di transizione per un \textit{NFA}}
\label{img:nfa}
\end{figure}

L'immagine \ref{img:nfa} mostra l'NFA $N=(\{1,2,3,4\},\{a,b\},\delta,0,\{3\})$
ed $\delta$ è possibile rappresentarlo mediante la seguente
\textbf{tabella di transizione} (\textit{transition table}):

\begin{table}[H]
\begin{center}
\begin{tabular}{|c|c|c|}
\hline
State & Input \textbf{a} & Input \textbf{b} \\ \hline
0 & \{0,1\} & \{0\} \\ \hline
1 & & \{2\} \\ \hline
2 & & \{3\} \\ \hline
\end{tabular}
\end{center}
\caption{Tabella di transizione}
\label{tab:transition_table}
\end{table}

equivalente a:
\begin{align*}
\delta(0,\textbf{a}) &= \{0,1\} \\
\delta(0,\textbf{b}) &= \{0\}   \\
\delta(1,\textbf{b}) &= \{2\}   \\
\delta(2,\textbf{b}) &= \{3\}
\end{align*}

\subsubsection{Linguaggio accettato da un NFA}
\begin{definition}[Language defined by a NFA]
Il linguaggio definito da un NFA $A$ è l'insieme di stringhe in input accettate,
denotato $L(A)$.
\end{definition}

Le stringe (su $\Sigma$) accettate da un NFA sono tutte le stringhe $w$ in cui
esiste un cammino del NFA con vertici etichettati con i simboli di $w$ in
sequenza\footnote{le $\epsilon$-transizioni non contribuisco come simboli} dallo
stato iniziale a qualche stato finale.

La transizione da uno stato all'altro sul grafo di transizione è detto
\textit{move}.

\subsubsection{Costruzione di Thompson}
\label{sec:costruzione_thompson}
La \textit{Thompson's construction} prende una RE e costruisce tramite induzione
strutturale un NFA che:
\begin{itemize}
\item \textbf{accetta esattamente lo stesso linguaggio dell'RE}
\item ha un singolo stato d'accettazione
\item nessuna transizione verso stato iniziale (no archi entranti)
\item nessuna transizione dallo stato finale (no archi uscenti)
\end{itemize}

La complessità è \textbf{lineare} all'espressione regolare. Per le regole si
veda
\url{https://www.di.unipi.it/~andrea/Didattica/PLP-16/SLIDES/PLP-2016-06.pdf}
slide \#10.

Un sistema semplice per unire più grafi è mediante le $\epsilon$-transizioni.
Si crea un nuovo spigolo come stato iniziale in cui partono tante
$\epsilon$-transizioni \textbf{tanti quanti} sono i grafi da collegare.

\subsection{RE a NFA a DFA}
L'uso di un DFA per l'accettazione del linguaggio è molto più efficiente. Il
problema è che se da RE a NFA la complessità era lineare la conversione da NFA
a DFA può risultare esponenziale con il numero di stati dell'NFA.

\subsubsection{Automa deterministico a stati finiti}
Un DFA è un caso speciale di NFA dove:
\begin{itemize}
\item non vi sono $\epsilon$-transizioni
\item $\forall s \in S \text{ e } a \in \Sigma \ \exists$ un vertice etichettato
$a$ uscente da $s$
\item ogni entry nella tabella di transizione è uno stato, esiste
\textbf{al più} un percorso per accettare una stringa
\end{itemize}

\subsubsection{Conversione di un NFA a DFA}
\label{sec:NFA_to_DFA}
L'algortimo per la conversione dell'NFA al DFA è chiamato
\textit{subset construction algorithm}, utilizzando due operatori:

\paragraph{$\epsilon$-closure($s$)} = $\{s\} \cup \{t \mid s \to_\epsilon \dots
\to_\epsilon t\}$
\paragraph{$\epsilon$-closure($T$)} = $\bigcup_{s \in T}
\epsilon\text{-closure}(s)$
\paragraph{move($T,a$)} = $\{t \mid s \to_a \text{ e } s \in T\}$

\bigskip

producento:
\begin{itemize}
\item \textit{Dstate} insieme degli stati del nuovo DFA considerando gli stati
dell'NFA
\item \textit{Dtran} tabella di transizione del nuovo DFA
\end{itemize}

Attraverso il medesimo algoritmo è possibile \textbf{simulare un NFA}
\begin{align*}
& S := \epsilon\text{-closure}(\{s_0\})                        \\
& S_{prev} := \emptyset                                        \\
& a := nextchar()                                              \\
& \mathbf{while} \ S \neq \emptyset \ \mathbf{do}              \\
& \quad S_{prev} := S                                          \\
& \quad S := \epsilon\text{-closure}(move(S,a))                \\
& \quad a := nextchar()                                        \\
& \mathbf{end \ do}                                            \\
& \mathbf{if} \ S_{prev} \cap F \neq \emptyset \ \mathbf{then} \\
& \quad \mathbf{execute} \ action \ in \ S_{prev}              \\
& \quad \mathbf{return} \ ``yes''                              \\
& \mathbf{else \ return} \ ``no''
\end{align*}

Il \textit{Subset Construction Algorithm} può essere cosi formulato:
\begin{align*}
& \mathbf{while} \ \text{there is a unmarked state} \ T \in Dstates      \\
& \quad mark \ T                                                         \\
& \quad \forall a \in \Sigma \ \mathbf{do}                               \\
& \qquad U := \epsilon\text{-closure}(move(T,a))                         \\
& \qquad \mathbf{if} \ U \notin Dstates \ \mathbf{then}                  \\
& \qquad \quad \text{add} \ U \ \text{as an unmarked state to} \ Dstates \\
& \qquad \mathbf{end \ if}                                               \\
& \qquad Dtran[T,a] := U                                                 \\
& \quad \mathbf{end \ do}                                                \\
& \mathbf{end \ do}
\end{align*}

\subsubsection{Minimizzazione stati DFA}
Una volta il DFA equivalente potrebbe essere possibile convertirlo in un altro
DFA il quale accetta lo stesso linguaggio regolare con il minimo numero di
stati. Ciò porterebbe benefici sia in termini di memoria che prestazioni.

\begin{definition}[Equivalent states]
Due stati $q$ e $q'$ in un DFA $M = (Q,\Sigma,\delta,q_0,F)$ sono equivalenti
(o indistinguibili) se: $\forall w \in \Sigma^*$ gli stati sul quale
$w$ termina, a partire da $q$ e $q'$, sono entrambi accetti o non-accetti.
\end{definition}

\begin{definition}[Irreducible automata]
Un automata si dice irriducibile se:
\begin{itemize}
\item non contiene inutili (irragiungibili) stati
\item non ci sono due stati distinti equivalenti
\end{itemize}
\end{definition}
\bigskip
L'algoritmo di minimizzazione (\textit{The Minimization Algorithm}) crea un
automa irriducibile il cui accetta lo stesso linguaggio.

Il processo di minimizzazione viene anche chiamato
\textit{Partition-refinement}: parte con la partizione degli stati
\{Accettanti, Non-accettanti\} e raffina fino alla conclusione.

\begin{align*}
& DFA \ \mathbf{minimize}(DFA(Q,\Sigma,d,q_0,F))                            \\
& \quad Q := Q - unreachableStateFrom(q_0)                                  \\
& \quad var \ P := \{F,Q-F\} \text{ // Partition }                          \\
& \quad var \ Consistent := false                                           \\
& \quad \mathbf{while} \ (Consisten = false)                                \\
& \qquad Consistent := true                                                 \\
& \qquad \forall \ (S \in P, a \in \Sigma, T \in P)                         \\
& \qquad \quad \text{// collect state of T that reach S using a}            \\
& \qquad \quad var \ temp := \{q \in T \mid d(q,a) \in S\}                  \\
& \qquad \quad \mathbf{if} \ (temp \neq \emptyset \wedge temp \neq T)       \\
& \qquad \qquad Consistent := false                                         \\
& \qquad \qquad P := (P \setminus {T}) \cup \{temp,T-temp\}                 \\
& \quad \mathbf{return} \ defineMinimizor((Q,\Sigma,d,q_0,F),P)
\end{align*}

\begin{align*}
& DFA \ \mathbf{defineMinimizor(DFA(Q,\Sigma,\delta,q_0,F), P)}             \\
& \quad var \ Q' := P                                                       \\
& \quad \text{// The set with } q_0 \text{. This is unique!}                \\
& \quad var \ q'_0 := \{C \in P \mid C \cap \{q_0\} \neq \emptyset\}        \\
& \quad var \ F' := \{S \in P \mid S \subseteq F\}                          \\
& \quad \forall \ (S \in P, a \in \Sigma)                                   \\
& \qquad var \ \delta'(S,a):=\{T\in P\mid\delta(s,a)\in T\ \forall s\in S\} \\
& \quad \mathbf{return} \ (Q',\Sigma,\delta',q'_0,F')
\end{align*}

Questo algoritmo oltre a garantire di produrre un grafo \textbf{irriducibile} è
anche il più piccolo possibile, dato il teorema seguente:

\begin{theorem}[THM - Myhill-Nerode]
  Prova
\end{theorem}
