\section{Lezione 2016-10-03}
\subsection{TODO}
% Insert what you need. Any row is associated with the improvment or mistake
% arise. In the first column you can insert what you should resolve or change,
% instead in the second column you may put the section where to apply some
% modification.
\begin{table}[ht]
\begin{center}
\begin{tabular}{|p{\textwidth}|c|}
\hline
\multicolumn{1}{|c|}{\textbf{Miglioramento}} & \textbf{Sezione} \\ \hline
Chiedere specifiche perchè realizzare un generatore di Lexical Analyzer
& \ref{sec:lexer_generatore} \\ \hline
Risolvere conflitti cosa intente? & \ref{sec:lexer_generatore} \\ \hline
\end{tabular}
\end{center}
\caption{Tabella miglioramenti}
\label{tab:tab_todo}
\end{table}

\subsection{Progettazione di un generatore di lexer}
\label{sec:lexer_generatore}
La progettazione di lexer che dato in input un linguaggio produce i token per il
parse richiede una serie di passaggi:
\begin{enumerate}
\item Dalla RE (\textit{regular expression}) si costruisce un NFA
(non-deterministic finite automaton) che accetta lo stesso linguaggio regolare
\item Combinare gli NFA in uno singolo
\item E' possibile \textbf{simulare direttamente} l'NFA o una volta determinato
l'NFA simulare il DFA (deterministic finite automaton) risultante
\item Risolvere conflitti
\end{enumerate}

\subsection{Non-deterministic Finite Automata}
\begin{definition}[Non-deterministic Finite Automata - NFA]
Un NFA è una 5-tupla $(S,\Sigma,\delta,s_0,F)$ dove:
\begin{itemize}
\item S è un insieme finito di stati
\item $\Sigma$ è un insieme finito di simboli, l'alfabeto
\item $\delta$ è una mappa da $S \times (\Sigma \cup \{\epsilon\})$ ad
un'insieme di stati
$$\delta : S \times (\Sigma \cup \{\epsilon\}) \to 2^S$$
\item $s_0 \in S$ è lo stato di partenza
\item $F \subseteq S$ è l'insieme degli stati accettanti (o finali)
\end{itemize}
\end{definition}
