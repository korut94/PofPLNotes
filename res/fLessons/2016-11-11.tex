\section{Lezione 2016-11-11}
\subsection{TODO}
% Insert what you need. Any row is associated with the improvment or mistake
% arise. In the first column you can insert what you should resolve or change,
% instead in the second column you may put the section where to apply some
% modification.
\begin{table}[H]
\begin{center}
\begin{tabular}{|p{\textwidth}|c|}
\hline
\multicolumn{1}{|c|}{\textbf{Miglioramento}} & \textbf{Sezione} \\ \hline
\end{tabular}
\end{center}
\caption{Tabella miglioramenti}
\label{tab:tab_todo}
\end{table}

\subsection{Liveness/Next-Use}
Trought a graph we can keep track of the variable's life. Instead keep the
information in a single block (scope), we can find the life for the whole
program. Also, we keep the local life information into the nodes.

\subsection{Dataflow analysis}
Many static analyses are dataflow analysis:
\begin{itemize}
\item Livenss
\end{itemize}

Step:
\begin{enumerate}
\item Determine the leaders: fist statement, each targer of the goto. So there
are the first istruction of the block and the immediatly next
\item Take the block and plug each leaders
\end{enumerate}

How many registers do I need to compile this program?
we considere $v$ live at a point of the program so there is a path from $v$
an another program $p$ where $v$ is used.

In this way we can calculate when a variable alive. In the control flow graph
attach in the edges the set of alive variables.

\subsection{Determining Live Sets}
The approch of the data flow analysis:
\begin{itemize}
\item consider the origin of information, where liveness holds because of immediate
observation
\item the propagation of informazion
\item joining information: how liveness info is passed acrosss the boundaries
of the block
\end{itemize}

\subsubsection{Orign information}
If a variable is used at a program point p, then it must be alvie immediately
before p.

If the variable is not used before p, we must see the propagatation:
\begin{enumerate}
\item it is alive immediate after p
\item it is not refefined at p
\end{enumerate}
