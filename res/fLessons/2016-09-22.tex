\section{Lesson 2016-09-22}
\subsection{Programming languages}
A \textbf{programming languages (PL)} is defined by:
\[syntax + semantics + pragmatics\]

\paragraph{syntax}
concerned with the \textbf{form of programs} through the constructs to make a
well-formed program.

\paragraph{semantics}
concerned with the \textbf{meaning of programs}. In other word, it's the
well-formed program's behavior when executed on a computer.

\paragraph{pragmatics}
concerned with the \textbf{way to use the programs} in practice. Pragmatics
include the \textit{paradigm(s)} supported by PL.

\subsection{Paradigms}
A paradigm is a \textbf{style of programming}. It is characterized by a set of
key concepts:

\begin{itemize}
\item \textbf{Imperative} - variables, commands, procedures
\item \textbf{Object-oriented} - objects, methods, classes
\item \textbf{Concurrent} - processes, communication
\item \textbf{Functional} - values, expressions, functions
\item \textbf{Logic} - assertions, relations
\end{itemize}

\subsection{Implementation of a Programming Language}
We call \(L\) the programming languages to make. L must be executable in
\textbf{Abstract Machine} \(M_L\) having L as \textbf{Machine Language} (or
\(L_M\)) implemented over a host machine \(M_O\) via \textbf{compilation},
\textbf{interpretation} or both.

\subsubsection{Abstract Machine}
Given the programming language \(L\), we can say that:
\begin{definition}
An \textbf{abstract machine} for \(L\) is a collection of data structures and
algorithms which can perform the storage and execution of programs written in
\(L\).
\end{definition}

This abstract machine \(M_L\) can be implemented in \textbf{hardware} or
\textbf{firmware}, over a machine \(M_O\) already implemented, through the
data structures and the algorithms implemented in the machine language of the
host.

There are two cases:
\begin{itemize}
\item interpreter of \(M_L =\) interpreter of \(M_O \Rightarrow\)
\textbf{extension} \textit{(of \(M_O\))}
\item interpreter of \(M_L \neq\) interpreter of \(M_O \Rightarrow\)
\textbf{interpreted} \textit{(over \(M_O\))}
\end{itemize}

other components of the machines may coincide or differ.

\subsubsection{Machine Language}
Given the abstract machine \(M\) we say that:
\begin{definition}
An \textbf{machine language} \(L_M\) of \(M\) includes all programs which can be
executed by the interpreter of \(M\).
\end{definition}

where the programs are particular data on which the interpreter can act.

Every abstract machine has a \textbf{unique} machine language, instead a machine
language can have a \textbf{several} abstract machines.

\subsection{Implementing a Programming Language}
Say \(L\) the programming language to build, so we call \(M_L\) the abstract
machine for \(L\) and \(M_O\) the host machine. To make \(L\) executable,
\(M_L\) should be run in \(M_O\) and to do this there are three way:
\textbf{pure interpretation}, \textbf{pure compilation} or \textbf{mixing}.

\subsubsection{Pure Interpretation}
The interpretation defines the function:
$$I^{L_O}_L : (P^L \times D) \to D \mid I^{L_O}_L(P^L, input) =
P^L(input)$$

\begin{figure}[H]
\begin{center}
\includegraphics[scale=0.7]{res/image/interpreter}
\caption{Interpreter schema}
\label{fig:interpreter_schema}
\end{center}
\end{figure}

where $P^L$ is the program written in $L$ and $D$ is the set of
\textit{input/output}. The image above is an abstract on how to work the
interpreter.

\subsubsection{Pure [cross] Compilation}
The compiler from $L$ to $L_O$ defines the function:
$$C_{L,L_O}: P^L \to P^{L_O} \mid C_{L,L_O}(P^L) = Pc^{L_O} \Rightarrow
P^L(input) = Pc^{L_O}(input), \quad \forall input \in D$$

\begin{figure}[ht]
\begin{center}
\includegraphics[scale=0.6]{res/image/compiler}
\caption{Compiler schema}
\label{fig:compiler_schema}
\end{center}
\end{figure}

\subsection{Compilers versus Interpreters}
Interpretation facilitates interactive debugging and testing, ex:
\begin{itemize}
\item Variable can be inspected and modified by a user
\item Procedures can be invoked from the command line
\end{itemize}

The compiling is \textbf{more efficiently} than the interpreting:

\paragraph{Decision at compile time}
The compiler fix decision to \textbf{avoid} to generate code that makes this
decision at run time.

\begin{itemize}
\item \textbf{Type checking}
\item Static allocation
\item Static linking
\end{itemize}

\paragraph{Leads to better performance}
It's possible doing aggressive code optimization to exploit hardware features
and variables allocating without variable lookup at run time.

\subsection{Compilation + Interpretation}
All implementations of programming languages use both. At least:
\begin{itemize}
\item Compilation (= translation) from \textbf{external} to \textbf{internal}
representation\footnote{Cosa vuole dire?}
\item Interpretation for I/O operations\footnote{Intenda tipo i comandi da
terminale?}
\end{itemize}

We introduce an additional \textit{Intermediate Abstract Machine} $M_I$ with the
language $L_I$, where:
\begin{enumerate}
\item $L$ is \textbf{compiled} to a program in $L_I$
\item $L_I$ is \textbf{executed} by an interpreter for $M_I$
\end{enumerate}

\begin{figure}[H]
\begin{center}
\includegraphics[scale=0.5]{res/image/mixing}
\caption{
Mixing schema - Nella slide \url{
https://www.di.unipi.it/~andrea/Didattica/PLP-16/SLIDES/PLP-2016-02.pdf
} n. 19 l'immagine non \`e chiara
}
\label{fig:mixing_schema}
\end{center}
\end{figure}

Several language implementations adopt a compilation + interpretation schema,
where the Intermediate Abstract Machine is called \textbf{Virtual Machine}
(ex. Java - Java Virtual Machine).

\subsection{Bootstrapping}
First, it's necessary introduce the concept of compiler as \textbf{graphic
element} and of \textbf{compilers composition}.

\subsubsection{Composing Compilers}
Three languages involved in writing a compiler:
\begin{itemize}
\item Source Language $S$
\item Target Language $T$
\item Implementation Language $I$
\end{itemize}

The image below show the relationship between the three programs, also called
\textbf{T-Diagram}

\begin{figure}[ht]
\begin{center}
\includegraphics[scale=0.6]{res/image/t_diagram}
\caption{T-Diagram representation}
\label{t-diagram_representation}
\end{center}
\end{figure}

This involve that we have two cases:
\begin{enumerate}
\item if $I = T$ we have a \textbf{Host Compiler}
\item if $S$, $T$ and $I$ are all different we have a \textbf{Cross-Compiler}
\end{enumerate}

Now compiling a compiler\footnote{Compilando un compilatore??? Non riesco a
capire come si otterrebbe il nuovo linguaggio...} we get a new one: the result
is describe by \textbf{composing} T-diagrams. The most general shape of the
basic transformation is the following:

\begin{figure}[H]
\begin{center}
\includegraphics[scale=0.5]{res/image/composition}
\caption{T-Diagrams composition}
\label{t-diagrams_composition}
\end{center}
\end{figure}

Obviously we assume that the programs written $M$ are able to run.

Now introduce the concept of \textbf{Bootstrapping}:

\begin{definition}
Bootstrapping are techniques which use \textbf{partial/inefficient} compiler
versions to generate a \textbf{complete/better} ones.
\end{definition}

Often this technique is used to compile a translator programmed in its own
language, there are several reasons to do this\footnote{Da vedere nel libro di
testo o chiedere al prof.}, but also it's possible use it also to make a
compiler \textbf{portable} or \textbf{optimize} it.

\subsubsection{Bootstrapping to port a compiler}
A compiler is portable if it satisfied two criteria:
\begin{itemize}
\item \textbf{retargetability}
\item \textbf{rehostability}
\end{itemize}

\begin{definition}
A retargetable compiler is one that can be modified easily to generate code for
a new target language.
\end{definition}

\begin{definition}
A rehostable compiler is one that can be moved easily to run on a new machine
\end{definition}

A portable compiler is less efficient than a host compiler because we cannot
make any specific assumption about the target machine.

In the context that we have a host compiler/interpreter of $L$ for $M$ and we
wrote a compiler of $L$ for $N$ in language $L$ itself, we can use the following
steps to perform the port:
\begin{enumerate}
\item \textbf{From $L$ to $M$}: translate the compiler of $L$ in a compiler of
$M$ so the compiler is implemented with an other language.

\begin{figure}[H]
\begin{center}
\includegraphics[scale=0.5]{res/image/from_L_to_M}
\caption{Translate compiler from L to M by composition}
\label{from_L_to_M}
\end{center}
\end{figure}

\item \textbf{From $M$ to $N$}: finally we can apply the composition result to
the original compiler so to achieve at the goal\footnote{Ma perchè sta roba
dovrobbe rendere il codice portabile?}.
\begin{figure}[H]
\begin{center}
\includegraphics[scale=0.5]{res/image/from_M_to_N}
\caption{Translate compiler from M to N by composition}
\label{from_M_to_N}
\end{center}
\end{figure}
\end{enumerate}

\subsubsection{Bootstrapping to optimize a compiler}
The idea is start with a simple compiler (generating inefficient code) and use
the bootstrapping to improve its performance. Any version will be more
sophisticated than ones before. Take the example of wanting improve a compiler
from ADA to $M*$ written in ADA. We have a host compiler of ADA to $M$, so we
can compose the us compiler with the host compiler to get a new compiler of
$ADA$ to $M*$ written in $M$ (fig. \ref{fig:from_ADA_to_M}).
Now compiling another times the us compiler with what obtained from the
composition we will get the optimized compiler (fig. \ref{fig:from_ADA_to_Ms}).

\begin{figure}[H]
  \centering
  \begin{subfigure}[b]{\textwidth}
    \includegraphics[width=\textwidth]{res/image/from_ADA_to_M}
    \caption{From ADA to M}
    \label{fig:from_ADA_to_M}
  \end{subfigure}
  ~
  \begin{subfigure}[b]{\textwidth}
    \includegraphics[width=\textwidth]{res/image/from_ADA_to_Ms}
    \caption{From ADA to M*}
    \label{fig:from_ADA_to_Ms}
  \end{subfigure}
  \caption{Bootstrapping to optimize a compiler}
  \label{fig:bootstrapping_optimize}
\end{figure}
